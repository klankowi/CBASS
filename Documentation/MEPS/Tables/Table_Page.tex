\documentclass{article}

\usepackage{PRIMEarxiv}
\usepackage[utf8]{inputenc} 	% allow utf-8 input
\usepackage[T1]{fontenc}    	% use 8-bit T1 fonts
\usepackage{hyperref}       		% hyperlinks
\usepackage{url}            		% simple URL typesetting
\usepackage{booktabs}       	% professional-quality tables
\usepackage{amsfonts}       	% blackboard math symbols
\usepackage{nicefrac}       		% compact symbols for 1/2, etc.
\usepackage{microtype}      	% microtypography
\usepackage{fancyhdr}      	 	% header
\usepackage{graphicx}       	% graphics
\usepackage[left]{lineno}		% line numbers
\usepackage{adjustbox}		% rotate tables
\usepackage{siunitx}			% Units
\usepackage{multirow}			% Multiple table rows
\usepackage{float}
\graphicspath{{media/}}     	% organize your images and other figures under media/ folder

% Set citation style
%\usepackage{citation-style-language}
%\cslsetup{style = canadian-journal-of-fisheries-and-aquatic-sciences}
%\addbibresource{library.bib}

%Header
%\pagestyle{fancy}
%\thispagestyle{empty}
%\rhead{ \textit{ }} 

% Update your Headers here
%\fancyhead[LO]{Atlantic Cod spatial density}
  
%% Title
%\title{Supplementary Tables
%%%% Cite as
%%%% Update your official citation here when published 
%\thanks{\textit{\underline{Citation}}: 
%\textbf{Authors. Title. Pages.... DOI:000000/11111.}} 
%}

\begin{document}

\begin{table}[]
\centering
\caption{Summary of seine catch data of 11 year time series. Species are arranged by number of individuals caught, number of seine hauls where at least one individual of the species was encountered (Encounters), and number of years where at least one individual of the species was encountered (Years encountered). The percentage of total catch (number caught divided by 159,590 total organisms enumerated) and percentage of encounters (number of seine hauls seen divided by 659 total seine hauls) are also calculated.}
\label{table:species-table}
\begin{tabular}{ccccccc}
Common name &  Scientific name &  Individs. &  \begin{tabular}[c]{@{}c@{}}\% of \\ total catch\end{tabular} &  Encounters &  \begin{tabular}[c]{@{}c@{}}\% of \\ encounters\end{tabular} &
  \begin{tabular}[c]{@{}c@{}}Years\\ encountered\end{tabular} \\ \hline
Cunner &  \textit{Tautogolabrus adspersus} &  1 &  \textless 0.01 &  1 &  0.2 &  1 \\
\begin{tabular}[c]{@{}c@{}}Eastern Silvery \\ Minnow\end{tabular} &  \textit{Hybognathus regius} &
  1 &  \textless 0.01 &  1 &  0.2 &  1 \\
Spotted Hake &  \textit{Urophycis regia} &  1 &  \textless 0.01 &  1 &  0.2 &  1 \\
Summer Flounder &  \textit{Paralichthys dentatus} &  1 &  \textless 0.01 &  1 &  0.2 &  1 \\
Northern Puffer &  \textit{Sphoeroides maculatus} &  2 &  \textless 0.01 &  2 &  0.3 &  2 \\
Red Hake &  \textit{Urophycis chuss} &  2 &  \textless 0.01 &  2 &  0.3 &  2 \\
American Eel &  \textit{Anguilla rostrata} &  3 &  \textless 0.01 &  3 &  0.5 &  3 \\
Crevalle Jack &  \textit{Caranx hippos} &  3 &  \textless 0.01 &  2 &  0.3 &  2 \\
Lumpfish &  \textit{Cyclopterus lumpus} &  3 &  \textless 0.01 &  2 &  0.3 &  2 \\
American Plaice &  \textit{\begin{tabular}[c]{@{}c@{}}Hippoglossoides\\ platessoides\end{tabular}} &
  4 &  \textless 0.01 &  1 &  0.2 &  1 \\
Shortfin Squid &  \textit{Illex illecebrosus} &  4 &  \textless 0.01 &  1 &  0.2 &  1 \\
Atlantic Butterfish &  \textit{Peprilus triacanthus} &  5 &  \textless 0.01 &  4 &  0.6 &  4 \\
Rainbow Smelt &  \textit{Osmerus mordax} &  5 &  \textless 0.01 &  3 &  0.5 &  2 \\
Striped Bass &  \textit{Morone saxatilis} &  6 &  \textless 0.01 &  6 &  0.9 &  3 \\
Smallmouth Bass &  \textit{Micropterus dolomieu} &  7 &  \textless 0.01 &  3 &  0.5 &  2 \\
White Hake &  \textit{Urophycis tenuis} &  7 &  \textless 0.01 &  2 &  0.3 &  2 \\
American Shad &  \textit{Alosa sapidissima} &  8 &  0.01 &  4 &  0.6 &  4 \\
Largemouth Bass &  \textit{Micropterus salmoides} &  9 &  0.01 &  4 &  0.6 &  3 \\
Permit &  \textit{Trachinotus falcatus} &  9 &  0.01 &  2 &  0.3 &  1 \\
Atlantic Cod &  \textit{Gadus morhua} &  10 &  0.01 &  2 &  0.3 &  1 \\
White Sucker &  \textit{Catostomus commersonii} &  19 &  0.01 &  6 &  0.9 &  5 \\
Blueback Herring &  \textit{Alosa aestivalis} &  20 &  0.01 &  6 &  0.9 &  3 \\
Longhorn Sculpin &  \textit{\begin{tabular}[c]{@{}c@{}}Myoxocephalus\\ octodecemspinosus\end{tabular}} &  20 &  0.01 &  7 &  1.1 &  2 \\
\begin{tabular}[c]{@{}c@{}}Threespine \\ Stickleback\end{tabular} &  \textit{Gasterosteus aculeatus} &
  21 &  0.01 &  15 &  2.3 &  7 \\
Rock Gunnel &  \textit{Pholis gunnellus} &  25 &  0.02 &  11 &  1.7 &  6 \\
Fallfish &  \textit{Semotilus corporalis} &  28 &  0.02 &  3 &  0.5 &  1 \\
Pollock &  \textit{Pollachius virens} &  35 &  0.02 &  6 &  0.9 &  5 \\
\begin{tabular}[c]{@{}c@{}}Ninespine \\ Stickleback\end{tabular} &  \textit{Pungitius pungitius} &
  42 &  0.03 &  6 &  0.9 &  4 \\
Shorthorn Sculpin &  \textit{Myoxocephalus scorpius} &  47 &  0.03 &  21 &  3.2 &  6 \\
Northern Pipefish &  \textit{Syngnathus fuscus} &  82 &  0.05 &  41 &  6.2 &  10 \\
Grubby Sculpin &  \textit{Myoxocephalus aenaeus} &  83 &  0.05 &  32 &  4.9 &  6 \\
Bluefish &  \textit{Pomatomus saltatrix} &  84 &  0.05 &  21 &  3.2 &  8 \\
White Mullet &  \textit{Mugil curema} &  180 &  0.11 &  9 &  1.4 &  6 \\
Atlantic Tomcod &  \textit{Microgadus tomcod} &  231 &  0.15 &  70 &  10.6 &  10 \\
Atlantic Menhaden &  \textit{Brevoortia tyrannus} &  682 &  0.43 &  4 &  0.6 &  3 \\
Winter Flounder &  \textit{\begin{tabular}[c]{@{}c@{}}Pseudopleuronectes\\ americanus\end{tabular}} &  1331 &  0.83 &  247 &  37.5 &  10 \\
Green Crab &  \textit{Carcinus maenas} &  4463 &  2.80 &  446 &  67.7 &  10 \\
Sandlance &  \textit{Ammodytes americanus} &  4688 &  2.94 &  65 &  9.9 &  9 \\
Alewife &  \textit{Alosa pseudoharengus} &  10184 &  6.38 &  140 &  21.2 &  10 \\
Mummichog &  \textit{Fundulus heteroclitus} &  14579 &  9.14 &  199 &  30.2 &  10 \\
Atlantic Herring &  \textit{Clupea harengus} &  55768 &  34.95 &  97 &  14.7 &  10 \\
Atlantic Silverside &  \textit{Menidia menidia} &  66887 &  41.91 &  369 &  56.0 &  10 \\ \hline
\end{tabular}
\end{table}

\begin{table}[h!]
\centering
\caption{Portland Harbor tide gauge temperature anomalies presented as the difference between each year’s average temperature and the expected annual temperature as calculated from the 2003-2020 climate reference period (CRP). Anomalies are also presented at a seasonal scale, with winter referring to December-February, spring referring to March-May, summer referring to June-August, and fall referring to September-November. Negative values are cooler temperatures than expected compared to the CRP, and positive values are warmer temperatures than expected compared to the CRP.}
\label{table:temp-anom}
\begin{tabular}{cccccc}
\multirow{2}{*}{Year} & \multirow{2}{*}{\begin{tabular}[c]{@{}c@{}}Annual Temp. \\ Anom. (\textdegree C)\end{tabular}} & \multicolumn{4}{c}{Seasonal Temp. Anom. (\textdegree C)} \\
     &       & Winter & Spring & Summer & Fall  \\
\hline
2014 & -0.74 & -1.28  & -1.13  & -0.62  & 0.06  \\
2015 & -0.41 & -0.74  & -1.18  & 0.06   & 0.21  \\
2016 & 0.69  & 1.34   & 0.26   & 0.57   & 0.6   \\
2017 & -0.37 & 0.3    & -0.74  & -1.05  & 0.05  \\
2018 & -0.11 & -0.84  & -0.06  & 0.58   & -0.13 \\
2019 & -0.27 & -0.78  & -0.41  & 0.35   & -0.24 \\
2020 & 0.66  & 0.21   & 0.62   & 1.86   & -0.08 \\
2021 & 1.24  & 0.32   & 1.15   & 1.7    & 1.76  \\
2022 & 0.97  & 0.28   & 0.97   & 1.51   & 1.1   \\
2023 & 1.12  & 1.04   & 0.97   & 1.42   & 1.03  \\
2024 & 0.9   & 0.91   & 0.95   & 1.09   & 0.64  \\ \hline
\end{tabular}
\end{table}

\end{document}
